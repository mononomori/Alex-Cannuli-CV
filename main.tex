
\PassOptionsToPackage{dvipsnames}{xcolor}
\documentclass[10pt,a4paper,ragged2e,withhyper]{altacv}
\newenvironment{sloppypar*}{\sloppy\ignorespaces}{\par}

% Page layout
\geometry{left=1.2cm,right=1.2cm,top=1cm,bottom=1cm,columnsep=0.75cm}

\usepackage{paracol}
\usepackage{enumitem}
\setlist{leftmargin=1.5em}

\ifxetexorluatex
  % If using xelatex or lualatex:
  \setmainfont{helvetica}
  \setsansfont{helvetica}
  \renewcommand{\familydefault}{\sfdefault}
\else
  % If using pdflatex:
  \usepackage[rm]{roboto}
  \renewcommand{\familydefault}{\sfdefault}
\fi

\ifdarkmode%
  \definecolor{PrimaryColor}{HTML}{C69749}
  \definecolor{SecondaryColor}{HTML}{D49B5}
  \definecolor{ThirdColor}{HTML}{1877E8}
  \definecolor{BodyColor}{HTML}{ABABAB}
  \definecolor{EmphasisColor}{HTML}{ABABAB}
  \definecolor{BackgroundColor}{HTML}{FFFFFF}
\else%
  \definecolor{PrimaryColor}{HTML}{001F5A}
  \definecolor{SecondaryColor}{HTML}{0039AC}
  \definecolor{ThirdColor}{HTML}{F3890B}
  \definecolor{BodyColor}{HTML}{4c4c4c}
  \definecolor{EmphasisColor}{HTML}{2E2E2E}
  \definecolor{BackgroundColor}{HTML}{f4f4f4}
\fi%

\colorlet{name}{PrimaryColor}
\colorlet{tagline}{SecondaryColor}
\colorlet{heading}{PrimaryColor}
\colorlet{headingrule}{ThirdColor}
\colorlet{subheading}{SecondaryColor}
\colorlet{accent}{SecondaryColor}
\colorlet{emphasis}{EmphasisColor}
\colorlet{body}{BodyColor}
\pagecolor{BackgroundColor}

\renewcommand{\namefont}{\Huge\rmfamily\bfseries}
\renewcommand{\personalinfofont}{\small\bfseries}
\renewcommand{\cvsectionfont}{\LARGE\rmfamily\bfseries}
\renewcommand{\cvsubsectionfont}{\large\bfseries}

\renewcommand{\itemmarker}{{\small\textbullet}}
\renewcommand{\ratingmarker}{\faCircle}
\newcommand*\sepline{%
  \begin{center}
    \rule[1ex]{.5\textwidth}{.5pt}
  \end{center}}

\begin{document}
    \name{Alexander Cannuli}
    \tagline{Software Developer}
    \photoL{4.5cm}{alex_logo small}
    \personalinfo{
        \email{mac42@sfu.ca}\email{alex.croft.cannuli@gmail.com}
        \phone{(604)-313-5342}\\
        \medskip
        \location{North Vancouver, Canada}
        \linkedin{alex-croft-cannuli}
        \github{mononomori}
        %\homepage{your_website}
        % \gitlab{your_id}
    }
    \makecvheader
    
    \columnratio{0.25}
    \begin{paracol}{2}
        \begingroup
        \vspace{4em}
        % ----- SKILLS -----
        \cvsection{Skills} 
           \cvlang{Languages}{JavaScript, React, Python, C\#, C/C++, Java, CSS/HTML, Nix}\\
            \divider
            \cvlang{Software}{Unity, Valve Hammer Editor, Photoshop, Lightroom}\\
            \divider
            \cvlang{OS}{Windows (7,10,11), Linux (NixOS, Mint, Ubuntu)}\\
            \divider
            \cvlang{Other}{Communicative, Adaptable, Level-headed, Task Management}\\
            \bigskip
        % ----- SKILLS -----
        
        % ----- INTERESTS -----
        \cvsection{Interests}
            \cvlang{Technical}{\\Video Games,\\Multimedia Processing,\\Glitch Art,\\Interactive Media,\\Software Art,\\Mechanical Keyboards,\\Retro-Technology}\\
            \divider{}
            \cvlang{Other}{\\Photography,\\Experimental Music,\\Science Fiction,\\Japanese Literature,\\Rock Climbing,\\Coffee}
            \bigskip
        % ----- INTERESTS ----- 
        \endgroup
    
        \newpage
        
        \switchcolumn
        
        % ----- TECHNICAL PROJECTS -----
        \vspace{-2em}
        \cvsection{Technical Project Experience}
            \cvschoolevent{\cvtitlereference{\faGithub harmonIQ}{https://github.com/ChristianChidiac/harmonIQ}}{Java, Springboot, and HTML/CSS}{May 2024 - Jul 2024} {CMPT 276 - Introduction to \medskip Software Engineering}
            \vspace{-0.5em}
            \begin{itemize}
            \item Developed a full-stack CRUD application that generates music-based quizzes by aggregating users’ Spotify listening history.
            \item Collaborated iteratively with team members through all stages of development, including planning, prototyping, and implementing, ensuring successful on-time delivery of all application deliverables.
            \end{itemize}
            \vspace{-0.5em}
            \divider
            \vspace{0.5em}
            \cvschoolevent{\cvtitlereference{\faGithub Fast InvSqrt() Comparative Analysis}{https://github.com/mononomori/Fast-invSqrt-Analysis}}{C/C++ and X86-64 Assembly}{Jun 2023 -- Aug 2023}{CMPT 295 - Introduction to Computer Systems}
            \vspace{-0.5em}
            \begin{itemize}
            \item Composed a comprehensive comparative analysis of the famous InvSqrt() algorithm employed in Quake III, evaluating its runtime performance and assembly output against both contemporary C/C++ implementations and x86-64 assembly instructions.
            \item Utilized Google's Benchmark library to conduct a detailed analysis of each functions CPU time and performance, quantifying individual assembly instruction's impact on resource utilization.
            \end{itemize}
            \vspace{-0.5em}
            \divider
            \vspace{0.5em}
            \cvschoolevent{\cvtitlereference{\faGithub N-Tile Puzzle Game Solver}{https://github.com/mononomori/N-tile-Puzzle-Game-Solver}}{Java}{Feb 2023 -- Apr 2023}{CMPT 225 - Data Structures and Programming}     
            \vspace{-0.5em}
            \begin{itemize}
            \item Developed a Java-based solver for the n-tile puzzle game, employing an understanding of heuristic and path finding optimization techniques to achieve sub-1-minute solve times for up to 6x6 boards.
            \item Evaluated the memory and CPU efficiency of various pathfinding algorithms, such as Dijkstra, A*, and IDA*, to find a balanced solution aligning with the project's requirements.
            \end{itemize}
        % ----- TECHNICAL PROJECTS -----
        
        % ----- PERSONAL PROJECTS -----
        \vspace{-1em}
        \cvsection{Personal Project Experience}
            \cvevent{Lead Programmer - Video Game}{SugarHIGH - Unreal (Game Engine)}{Mar 2024 -- Mar 2024}{\hspace{-0.5em}\cvreference{\faGlobe}{https://slurmsmckenzie.itch.io/sugarhigh} | \cvreference{\faGithub}{https://github.com/mononomori/SugarCube}}
            \vspace{-0.5em}
            \begin{itemize}
            \item Served as Lead Programmer for a 3D puzzle-platformer, our entry in the Buddy Up Jam: Winter 2024, successfully delivering a fully playable game within the 7 day deadline.
            \item Designed and implemented a custom physics-based movement system in Unreal Engine to create unique gameplay dynamics.
            \end{itemize}
            \vspace{-0.5em}
            \divider
            \vspace{0.5em}
            \cvevent{Lead Programmer - Video Game}{Daikon Run - Unity (Game Engine)}{Dec 2023 -- Feb 2024}{}
            \vspace{-0.5em}
            \begin{itemize}
            \item Worked as developer in a two-person team to create an original 2D side-scrolling auto-run game in Unity, coordinating with a designer to deliver a fully functional game with complete gameplay mechanics and a unique aesthetic.
            \end{itemize}
        \end{paracol}
        % ----- PERSONAL PROJECTS -----
        \newpage

        \makecvheader

        % ----- WORK EXPERIENCE -----
        \cvsection{Work Experience}
            \cvlocationevent{Shift Supervisor}{49th Parallel Café (Lonsdale / Main Street)}{Nov 2019 -- Current}{North Vancouver / Vancouver, Canada}
            \vspace{-0.5em}
            \begin{itemize}
            \item Supported a 10-person team during café opening, showcasing strong communication skills by explaining offerings to a new customer base while assisting newly trained front-of-house workers with cafe operations. 
            \item Maintained level-headed approach during high-pressure periods serving 200+ customers, multitasking across various responsibilities while adapting quickly to changing priorities to ensure smooth operations.
            \item Collaborated on an analysis of operational procedures and co-authored a comprehensive report of recommendations that helped eliminate recurring order processing errors and streamline management staff scheduling.
            \end{itemize}
            \vspace{-0.5em}
            \divider
            \vspace{0.5em}
            \cvlocationevent{Content Specialist}{ICUC - Marketing Services}{Sep 2014 -- Mar 2019}{Remote}
            \vspace{-0.5em}
            \begin{itemize}
            \item Showcased strong multitasking abilities by concurrently managing 30+ customer issues daily across multiple channels while maintaining consistent brand voice and delivering fast resolution times.
            \item Demonstrated strong teamwork by working with colleagues to develop standardized quality control processes for public-facing responses.
            \end{itemize}
        % ----- WORK EXPERIENCE -----

        % ----- EDUCATION -----
        \cvsection{Education}
            \cvlocationevent{BSc - Computing Science}{Simon Fraser University}{Nov 2023 -- Ongoing}{BC, Canada}
        \vspace{-1em}
        \cvlist{
          \begin{itemize}
            \item \textbf{CMPT 363 – User Interface Design:} Built a React-based UI prototype illustrating human-first and iterative design paradigms, conducted heuristic usability tests with users to improve upon app features.
            \item \textbf{CMPT 365 – Multimedia Systems:} Developed a Python-based image editing tool, implementing pixel manipulation and compression algorithms with various performance metrics.
            \item \textbf{CMPT 361 – Visual Computing:} Implemented interactive WebGL demos in JavaScript demonstrating polygon meshes, texturing, illumination, and shader programming.
          \end{itemize}
        }

            \divider
            \vspace{0.5em}
            \cvlocationevent{Certificate - Software Systems}{Vancouver Community College}{May 2021 - Aug 2022}{BC, Canada}
            \vspace{-1em}
        \cvlist{
            \begin{itemize}
            \item \textbf{CMPT 1010/1020 – Intro to Programming 1/2:} Developed foundational OOP skills in C/C++ and Python, demonstrated through creation of a datamoshing and pixel-shifting application.
          \end{itemize}
        }
        % ----- EDUCATION -----
        
\end{document}